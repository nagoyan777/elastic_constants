\documentclass[11pt,a4paper,uplatex]{jsarticle}
%
\usepackage{amsmath,amssymb,mathbbol}
\usepackage{bm}
\usepackage[dvipdfmx]{graphics,graphicx}
\usepackage{braket}
\usepackage{ascmac}
\usepackage{url}
\usepackage{newtxmath}
%
\DeclareMathOperator*{\ssum}{\Sigma} %小さい\sum
\setlength{\textwidth}{\fullwidth}
\setlength{\textheight}{39\baselineskip}
\addtolength{\textheight}{\topskip}
\setlength{\voffset}{-0.5in}
\setlength{\headsep}{0.3in}
\bibliographystyle{junsrt}

\newcommand{\divergence}{\mathrm{div}\,}  %ダイバージェンス
\newcommand{\grad}{\mathrm{grad}\,}  %グラディエント
\newcommand{\rot}{\mathrm{rot}\,}  %ローテーション
\newcommand{\vect}[1]{\boldsymbol{\mathbf{#1}}}
%
\begin{document}
\section{はじめに}
Lutskoによるストレスゆらぎと弾性定数の関係式の導出を示す\cite{Lutsko1989}。
Lutsko以前に、Rayらによる導出がある。
Rayらの導出ではハミルトニアンに圧力制御のピストンの運動エネルギー、及び、ポテンシャルが現れる。
一方、これらはLuskoの導出には現れない。Luskoの導出のほうがわかりやすいためこちらに従う。
\section{定義と関係式}
系のハミルトニアンは粒子の運動エネルギーとポテンシャルの和で表される。
\begin{align}
    {\cal H} = \sum_{i} \frac{p_{i}^{2}}{2m_{i}} + V(\{\bm{q}_{i}\})
\end{align}    
ここで、$i$番目の粒子の位置$\bm{q}_{i}$、及び、運動量$\bm{p}_{i}$はそれぞれ3次元の縦ベクトルとする。
\begin{align}
\bm{q} &= \begin{pmatrix} q_{1} \\ q_{2} \\ q_{3} \end{pmatrix}, \ \ 
\bm{p} = \begin{pmatrix} p_{1} \\ p_{2} \\ p_{3} \end{pmatrix} 
\end{align}
格子ベクトル$\bm{a}_{1}$、$\bm{a}_{2}$、$\bm{a}_{3}$を用いて格子行列${\rm H}$を定義する。
\begin{align}
{\rm H} &= \begin{pmatrix} \bm{a}_{1} & \bm{a}_{2} & \bm{a}_{3} \end{pmatrix} 
    = \begin{pmatrix} 
        a_{11} & a_{21} & a_{31} \\
        a_{12} & a_{22} & a_{32} \\
        a_{13} & a_{23} & a_{33} \\
    \end{pmatrix} 
\end{align}
これらを用いて、内部座標$\bm{s}$、及び、スケールされた運動量$\tilde{p}$を定義する。
\begin{align}
    \bm{s} &= {\rm H}^{-1} \bm{q}, \ \
    \bm{\tilde{p}} = {\rm H}^{T} \bm{p}
\end{align}
\begin{align}
    \bm{q} &= {\rm H} \bm{s} \\
    &= \sum_{i} s_{i} \bm{a}_{i} \\
    &=          \begin{pmatrix} 
        a_{11} & a_{21} & a_{31} \\
        a_{12} & a_{22} & a_{32} \\
        a_{13} & a_{23} & a_{33} \\
    \end{pmatrix} 
    \begin{pmatrix}
        s_{1} \\ s_{2} \\ s_{3} 
    \end{pmatrix} \\
    \bm{p} &= ({\rm H}^{-1})^{T} \bm{\tilde{p}} 
\end{align}  
以下では、表記を簡単にするために逆行列の転置を${\rm H}^{-T}:=\left({\rm H}^{-1}\right)^{T}=\left({\rm H}^{T}\right)^{-1}$と表す。
基準となる平衡状態の格子ベクトルを${\rm H}_{0}$として、応力により変形した格子ベクトルを${\rm H}={\rm H}_{0}+{\rm \delta H}$とする。これらを用いて歪みテンソル$e$を定義し、格子の変形を表す。
\begin{align}
    {\rm H} &= {\rm H_{0} + \delta H} \\
    {\rm H^{T} H} &= {\rm (H_{0} + \delta H)^{T} (H_{0} + \delta H)} \\
    &= {\rm H_{0}^{T} H_{0} + \delta H^{T} H_{0} + H_{0}^{T} \delta H + \delta H^{T} \delta H} \\
    &\approx {\rm H_{0}^{T} H_{0} + \delta H^{T} H_{0} + H_{0}^{T} \delta H } \\
    &:= {\rm H_{0}^{T} H_{0} + 2 H_{0}^{T} e H_{0}} \\
    &= {\rm H_{0}^T (1 + 2 e) H_{0}} \\
{\rm e} &:= {\rm \frac{1}{2}\left((H_{0}^{T})^{-1} H^{T} H H_{0}^{-1} - 1 \right) } \\
%{\rm H_{0}^{-T} } &:= {\rm (H_{0}^{T})^{-1}} = {\rm (H_{0}^{-1})^{T} }
\end{align}
巨視的なストレステンソル$t_{\alpha \beta}$、及び、弾性定数$C_{\alpha \beta \gamma \sigma}$はそれぞれヘルムホルツ自由エネルギー$F$、及び、ストレステンソル$t_{\alpha \beta}$の歪み$e$に対する微分として定義される。
\begin{align}
 t_{\alpha \beta} &= -\frac{\partial F}{\partial e_{\alpha \beta}}, \ \
 C_{\alpha \beta \gamma \sigma} = -\frac{\partial t_{\alpha \beta}}{e_{\gamma \beta}}
\end{align}
弾性定数$C_{\alpha \beta \gamma \sigma}$をストレスゆらぎで表すには、これらの微分の表式を求めれば良い。

\section{微分の表式}
任意のスカラー関数$A(\{\bm{q}_{i}, \bm{p}_{i}\})$の${\rm H}$、$e$についての微分の表式を求める。まず、変数$\bm{q}_{i}$、$\bm{p}_{i}$の微分を求める。
\begin{align}
    \frac{\partial q_{i \alpha}}{\partial H_{\beta \gamma}} 
    &= \frac{\partial}{\partial H_{\beta \gamma}} \sum_{\sigma} H_{\alpha \sigma} \bf{s}_{\sigma} \\
    &= \sum_{\sigma} \delta_{\alpha \beta} \delta_{\sigma \gamma} s_{\sigma} \\
    &= \delta_{\alpha \beta} s_{\gamma} \\
    &= \delta_{\alpha \beta} ({\rm H}^{-1} \bm{q})_{\gamma} \\
    &= \delta_{\alpha \beta} \sum_{\sigma} {\rm H}^{-1}_{\gamma \sigma} \bm{q}_{\sigma}
\end{align}
\begin{align}
    \frac{\partial \tilde{ p}_{i \alpha} }{\partial H_{\beta \gamma}} 
    &= \frac{\partial }{\partial H_{\beta \gamma}} \left({\rm H}^{T} \bf{p}_{i}\right)_{\alpha} \\
    0 &= \frac{\partial }{\partial H_{\beta \gamma}} \sum_{\sigma} H^{T}_{\alpha \sigma} p_{i \sigma}\\
    &= \sum_{\sigma} \left( \frac{\partial H_{\sigma \alpha} }{\partial H_{\beta \gamma}} p_{i \sigma}  +  H_{\sigma \alpha} \frac{\partial  p_{i \sigma}}{\partial H_{\beta \gamma}} \right) \\
    &= \sum_{\sigma} \left( \delta_{\sigma \beta} \delta_{\alpha \gamma} p_{i \sigma}  +  H_{\sigma \alpha} \frac{\partial  p_{i \sigma}}{\partial H_{\beta \gamma}} \right) \\
    &= \delta_{\alpha \gamma} p_{i \beta} + \sum_{\sigma} H_{\sigma \alpha} \frac{\partial  p_{i \sigma}}{\partial H_{\beta \gamma}}\\
    \sum_{\sigma} H_{\sigma \alpha} \frac{\partial  p_{i \sigma}}{\partial H_{\beta \gamma}} &= - \delta_{\alpha \gamma} p_{i \beta} 
    %frac{\partial }{\partial H_{\beta \gamma}} \left({\rm H}^{T} \bf{p}_{i}\right)_{\alpha} 
    \\
    \sum_{\alpha} H^{-1}_{\alpha \tau} \sum_{\sigma} H_{\sigma \alpha} \frac{\partial  p_{i \sigma}}{\partial H_{\beta \gamma}} &= - \sum_{\alpha} H^{-1}_{\alpha \tau}  \delta_{\alpha \gamma} p_{i \beta}  
    % + \sum_{\alpha} H^{-1}_{\alpha \tau} \frac{\partial }{\partial H_{\beta \gamma}} \left({\rm H}^{T} \bf{p}_{i}\right)_{\alpha} 
    \\
    \sum_{\sigma} \left( H H^{-1} \right)_{\sigma \tau}  \frac{\partial  p_{i \sigma}}{\partial H_{\beta \gamma}} &= - H^{-1}_{\gamma \tau} p_{i \beta}  \\% + \sum_{\alpha} H^{-1}_{\alpha \tau} \frac{\partial }{\partial H_{\beta \gamma}} \left({\rm H}^{T} \bf{p}_{i}\right)_{\alpha} \\
    \sum_{\sigma} \delta_{\sigma \tau}  \frac{\partial  p_{i \sigma}}{\partial H_{\beta \gamma}} &= - H^{-1}_{\gamma \tau} p_{i \beta}  \\% + \sum_{\alpha} H^{-1}_{\alpha \tau} \frac{\partial }{\partial H_{\beta \gamma}} \left({\rm H}^{T} \bf{p}_{i}\right)_{\alpha} \\
    \frac{\partial  p_{i \tau}}{\partial H_{\beta \gamma}} &= - H^{-1}_{\gamma \tau} p_{i \beta}  % + \sum_{\alpha} H^{-1}_{\alpha \tau} \frac{\partial }{\partial H_{\beta \gamma}} \left({\rm H}^{T} \bf{p}_{i}\right)_{\alpha}  \\
\end{align}
したがって、$q_{i\alpha}$、$p_{i\alpha}$の$H_{\beta \gamma}$微分は以下である。
\begin{align}
    \frac{\partial q_{i \alpha}}{\partial H_{\beta \gamma}} 
    &= \delta_{\alpha \beta} \sum_{\sigma} {\rm H}^{-1}_{\gamma \sigma} \bm{q}_{\sigma}, \ \
    \frac{\partial  p_{i \alpha}}{\partial H_{\beta \gamma}} = - H^{-1}_{\gamma \alpha} p_{i \beta}  
\end{align}
スカラー関数$A\left(\left\{ \bm{q}_{i}, \bm{p}_{i} \right\}\right)$の$H_{\beta \gamma}$微分は以下となる。
\begin{align}
    \frac{\partial {\rm A}(\{\bm{q}_{i}, \bm{p}_{i}\})}{\partial H_{\beta \gamma}}  &= \sum_{i, \alpha} \left\{ \frac{\partial {\rm A}(\{\bm{q}_{i}, \bm{p}_{i}\})}{\partial q_{i \alpha}} \frac{\partial q_{i \alpha}}{\partial H_{\beta \gamma}} 
    +  \frac{\partial {\rm A}(\{\bm{q}_{i}, \bm{p}_{i}\})}{\partial p_{i \alpha}} \frac{\partial p_{i \alpha}}{\partial H_{\beta \gamma}} \right\} \\
    &= \sum_{i} \left\{ \sum_{\alpha} \frac{\partial {\rm A}(\{\bm{q}_{i}, \bm{p}_{i}\})}{\partial q_{i \alpha}} \left( \delta_{\alpha \beta} \sum_{\sigma} {\rm H}^{-1}_{\gamma \sigma} q_{i \sigma} \right) + \sum_{\alpha} \frac{\partial {\rm A}(\{\bm{q}_{i}, \bm{p}_{i}\})}{\partial p_{i \alpha}} \left(-H_{\gamma \alpha}^{-1} p_{i\beta} \right) \right\} \\
    &= \sum_{i} \left\{ \frac{\partial {\rm A}(\{\bm{q}_{i}, \bm{p}_{i}\})}{\partial q_{i \beta}} \sum_{\sigma} {\rm H}^{-1}_{\gamma \sigma} q_{i \sigma} 
    -  p_{i\beta} \sum_{\alpha} H_{\gamma \alpha}^{-1} \frac{\partial {\rm A}(\{\bm{q}_{i}, \bm{p}_{i}\})}{\partial p_{i \alpha}} \right\} \\
    &= \sum_{i} \left\{ \frac{\partial {\rm A}(\{\bm{q}_{i}, \bm{p}_{i}\})}{\partial q_{i \beta}} \left({\rm H}^{-1} q_{i}\right)_{\gamma}
    -  p_{i \beta} \left( H^{-1} \frac{\partial {\rm A}(\{\bm{q}_{i}, \bm{p}_{i}\})}{\partial \bm{p}_{i}}  \right)_{\gamma} \right\}
\end{align}
${\rm H}$の各成分での微分を行列として表すと次式となる。
\begin{align}
    \frac{\partial A}{\partial {\rm H}} &= \sum_{i} \left\{
    \frac{\partial A}{\partial \bm{q}_{i}} \otimes \left({\rm H}^{-1} \bm{q}_{i}\right)
    -  \bm{p}_{i} \otimes \left( {\rm H^{-1}} \frac{\partial A}{\partial \bm{p}_{i}}  \right) 
    \right\}
\end{align}
ここで、$\otimes$はテンソル積を表す。例えば${\rm H}_{\alpha \beta}$微分の場合、$\otimes$の左側、右側はそれぞれ$\alpha$成分、$\beta$成分をとる。

歪みテンソル$e$での微分を計算するために、$e$の微小変化${\rm de}$に対する${\rm H}$の変化${\rm dH}$を求める。
\begin{align}
    {\rm H^{T} H} &= {\rm H_{0}^{T}} ( 1 + 2 {\rm e}) {\rm H_{0} } \\
    {\rm (H + dH)^{T} (H+dH)} &= {\rm H_{0}^{T}} (1 + 2({\rm e+de })) {\rm H_{0} } \\
    {\rm H^{T} dH + dH^{T} H + dH^{T} dH } &= 2 {\rm H_{0}^{T} de H_{0} }\\
    {\rm de} &= \frac{1}{2} {\rm H_{0}}^{-T} ( {\rm dH^{T} H + H^{T} dH } ) {\rm H_{0}^{-1}}
\end{align}
ここで、$de$は対称行列で表され、独立な成分は6個である。
対称行列で表される歪みテンソル$e$に対して、反対称テンソル${\rm dw}$を定義する。
\begin{align}
    {\rm dw} &:= \frac{1}{2} {\rm H_{0}}^{-T} ( {\rm  dH^{T} H - H^{T} dH } ) {\rm H_{0}^{-1}} \\
\end{align}
${\rm dw}$の独立な成分は3個である。したがって、$de$と$dw$とから、${\rm H}$と同様に9つの独立な変数が得られる。
\begin{align}
    {\rm de + dw} &= {\rm H_{0}}^{-T} {\rm dH^{T} H  H_{0}^{-1} } \\
    {\rm de - dw} &= {\rm H_{0}}^{-T} {\rm H^{T} dH  H_{0}^{-1} } \\
    {\rm dH} &= {\rm H^{-T} H_{0}^{T}} ({\rm de - dw}) {\rm H_{0}} \\
    {\rm dH}^{T} &= {\rm H_{0}^{T}}({\rm de + dw}) {\rm H_{0}} {\rm H}^{-1}\\
\end{align}

$A$の歪みに対する微分を求める。
\begin{align}
    dA({\bm{q}_{i},\bm{p}_{i}}) &= \sum_{\alpha,\beta} \frac{\partial A}{\partial H_{\alpha \beta}} dH_{\alpha \beta} \\
    &= \sum_{\alpha,\beta} \frac{\partial A}{\partial H_{\alpha \beta}} dH_{\beta \alpha}^{T} \\
    &= \mathrm{Tr} \left(\frac{\partial A}{\partial {\rm H}} {\rm dH}^{T} \right) \\
    &= \mathrm{Tr} \left(\frac{\partial A}{\partial {\rm H}} \left(  {\rm H_{0}^{T}}({\rm de + dw}) {\rm H_{0}} {\rm H}^{-1} \right) \right) \\
    &= \mathrm{Tr} \left({\rm H}_{0} {\rm H}^{-1} \frac{\partial A}{\partial {\rm H}} {\rm H}_{0}^{T} ({\rm de + dw}) \right) \\
    &= \sum_{\alpha \beta} \left({\rm H}_{0} {\rm H}^{-1} \frac{\partial A}{\partial {\rm H}} {\rm H}_{0}^{T} \right)_{\alpha \beta} ({\rm de + dw})^{T}_{\alpha \beta}\\
    &= \sum_{\alpha \beta} \left({\rm H}_{0} {\rm H}^{-1} \frac{\partial A}{\partial {\rm H}} {\rm H}_{0}^{T} \right)_{\alpha \beta} ({\rm de - dw})_{\alpha \beta}\\
\end{align}
\begin{align}
    dA({\bm{q}_{i},\bm{p}_{i}}) &= \sum_{\alpha,\beta} \frac{\partial A}{\partial H_{\alpha \beta}} dH_{\alpha \beta} \\
    &= \sum_{\alpha,\beta} \frac{\partial A}{\partial H_{\beta \alpha}^{T}} dH_{\alpha \beta} \\
    &= \mathrm{Tr} \left(\frac{\partial A}{\partial {\rm H^{T}}} {\rm dH} \right) \\
    &= \mathrm{Tr} \left(\frac{\partial A}{\partial {\rm H^{T}}} \left(  {\rm H^{-T} H_{0}^{T}} ({\rm de - dw}) {\rm H_{0}} \right) \right) \\
    &= \mathrm{Tr} \left({\rm H}_{0} \frac{\partial A}{\partial {\rm H^{T}}} {\rm H}^{-T}{\rm H}_{0}^{T} ({\rm de - dw}) \right) \\
    &= \sum_{\alpha \beta}\left({\rm H}_{0} \frac{\partial A}{\partial {\rm H^{T}}} {\rm H}^{-T}{\rm H}_{0}^{T}\right)_{\alpha \beta} ({\rm de - dw})_{\beta \alpha} \\
    &= \sum_{\alpha \beta}\left({\rm H}_{0} \frac{\partial A}{\partial {\rm H^{T}}} {\rm H}^{-T}{\rm H}_{0}^{T}\right)_{\alpha \beta} ({\rm de + dw})_{\alpha \beta} \\
\end{align}

\begin{align}
    {\rm dA} &= \sum_{\alpha \beta} \left( 
    \frac{1}{2}\left({\rm H}_{0} {\rm H}^{-1} \frac{\partial A}{\partial {\rm H}} {\rm H}_{0}^{T} + {\rm H}_{0} \frac{\partial A}{\partial {\rm H^{T}}} {\rm H}^{-T}{\rm H}_{0}^{T}  \right)_{\alpha \beta} {\rm de}_{\alpha \beta}
     + \frac{1}{2}\left({\rm H}_{0} {\rm H}^{-1} \frac{\partial A}{\partial {\rm H}} {\rm H}_{0}^{T} - {\rm H}_{0} \frac{\partial A}{\partial {\rm H^{T}}} {\rm H}^{-T}{\rm H}_{0}^{T}  \right)_{\alpha \beta} {\rm dw}_{\alpha \beta} 
     \right)
\end{align}
\begin{align}
    \frac{\partial {\cal H}}{\partial q_{i \alpha}} &= \frac{\partial}{\partial q_{i \alpha}} \left( \sum_{j} \sum_{\beta}\frac{p_{j \beta}^{2}}{2m_{j}} + V \left(\{\bm{q}_{j}\}\right)\right) 
    = \frac{\partial}{\partial q_{i \alpha}} V \left(\{\bm{q}_{j}\}\right) 
    := V_{i \alpha} \\
    \frac{\partial {\cal H}}{\partial p_{i \alpha}} &= \frac{\partial}{\partial p_{i \alpha}} \left( \sum_{j} \sum_{\beta}\frac{p_{j \beta}^{2}}{2m_{j}} + V \left(\{\bm{q}_{j}\}\right)\right) 
    = \frac{p_{i \alpha}}{m_{i}}
\end{align}
\begin{align}
    \frac{\partial {\cal H}}{\partial \bm{q}_{i}} &= \bm{V}_{i} ,  \ \ \ \  
    \frac{\partial {\cal H}}{\partial \bm{p}_{i}} = \frac{\bm{p}_{i}}{m_{i}}
\end{align}
\begin{align}
    \frac{\partial {\cal H}}{\partial H_{\alpha \beta}} 
    &= \sum_{i} \left\{ \frac{\partial {\cal H}}{\partial q_{i \alpha}} \left({\rm H}^{-1} q_{i}\right)_{\beta}
    -  p_{i \alpha} \left( H^{-1} \frac{\partial {\cal H}}{\partial \bm{p}_{i}}  \right)_{\beta} \right\} \\
    &= \sum_{i} \left\{ V_{i \alpha} \left({\rm H}^{-1} q_{i}\right)_{\beta}
    -  p_{i \alpha} \left( H^{-1}\frac{\bm{p}_{i}}{m_{i}}  \right)_{\beta} \right\} \\
    &= \sum_{i} \left\{ V_{i \alpha} \left({\rm H}^{-1} q_{i}\right)_{\beta}
    -  p_{i \alpha} \left( \frac{\bm{p}_{i}}{m_{i}} H^{-T} \right)_{\beta} \right\}  \\
    \frac{\partial {\cal H}}{\partial {\rm H}} &= \sum_{i}\left\{ \bm{V}_{i} \otimes   \left({\rm H}^{-1} \bm{q}_{i}\right) 
    - \bm{p}_{i} \otimes \left( H^{-1}\frac{\bm{p}_{i}}{m_{i}}  \right) \right\}     
\end{align}

\begin{align}
    \frac{\rm dA}{\rm de} &= \frac{1}{2}\left({\rm H}_{0} {\rm H}^{-1} \frac{\partial A}{\partial {\rm H}} {\rm H}_{0}^{T} + {\rm H}_{0} \frac{\partial A}{\partial {\rm H^{T}}} {\rm H}^{-T}{\rm H}_{0}^{T}  \right) \\
    \frac{\partial A}{\partial e_{\alpha \sigma}} &= \frac{1}{2} \sum_{i} \sum_{\beta \gamma} \left( \left({\rm H}_{0} {\rm H}^{-1}\right)_{\alpha \beta}
     \left\{ \frac{\partial {\rm A}}{\partial q_{i \beta}} \left({\rm H}^{-1} q_{i}\right)_{\gamma}    -  p_{i \beta} \left( H^{-1} \frac{\partial {\rm A}}{\partial \bm{p}_{i}}  \right)_{\gamma} \right\} {\rm H}_{0 \gamma \sigma} ^{T}\right. \\
    &+ \left. {\rm H}_{0 \alpha \beta}  \left\{ \left({\rm H}^{-1} q_{i}\right)_{\beta} \frac{\partial {\rm A}}{\partial q_{i \gamma}}  -  \left( H^{-1} \frac{\partial {\rm A}}{\partial \bm{p}_{i}}  \right)_{\beta} p_{i \gamma} \right\}   \left({\rm H}^{-T}{\rm H}_{0}^{T} \right)_{\gamma \sigma} \right) \\
\end{align}


\begin{align}
    \frac{\partial {\rm A}(\{\bm{q}_{i}, \bm{p}_{i}\})}{\partial H_{\beta \gamma}} 
    &= \sum_{i} \left\{ \frac{\partial {\rm A}(\{\bm{q}_{i}, \bm{p}_{i}\})}{\partial q_{i \beta}} \left({\rm H}^{-1} q_{i}\right)_{\gamma}
    -  p_{i \beta} \left( H^{-1} \frac{\partial {\rm A}(\{\bm{q}_{i}, \bm{p}_{i}\})}{\partial \bm{p}_{i}}  \right)_{\gamma} \right\} \\
    \frac{\partial A}{\partial {\rm H}} &= \sum_{i} \left\{ 
    \frac{\partial A }{\partial \bm{q}_{i}} \otimes \left({\rm H}^{-1} \bm{q}_{i} \right) 
   - \bm{p}_{i} \otimes \left( {\rm H}^{-1} \frac{\partial A }{\partial \bm{p}_{i}} \right)
    \right\}\\
    \frac{\partial A}{\partial {\rm H}^T} &= \sum_{i} \left\{\left({\rm H}^{-1} \bm{q}_{i} \right)  \otimes \frac{\partial A }{\partial \bm{q}_{i}} 
    - \left( {\rm H}^{-1} \frac{\partial A }{\partial \bm{p}_{i}} \right)\otimes \bm{p}_{i} 
    \right\}
\end{align}
\begin{align}
    \left( A \left(b \otimes c\right) \right)_{ik} &= \sum_{j} A_{ij} b_{j} c_{k}  
    = (Ab)_{i} c_{k} 
    = (b A^{T})_{i} c_{k} \\
    &= \left( (Ab) \otimes c \right)_{ik}\\
    &= \left( (bA^{T}) \otimes c \right)_{ik}\\
    \left( (a \otimes b) C\right)_{ik} &= \sum_{j} a_{i} b_{j} C_{jk} \\
    &= \left( a \otimes \left(b C\right)\right)_{ik} \\
    &= \left( a \otimes \left(C^{T} b \right)\right)_{ik}
\end{align}
\begin{align}
    {\rm H}^{-1} \frac{\partial A}{\partial {\rm H}} + \frac{\partial A}{\partial {\rm H}^{T}}{\rm H}^{-T} 
    &= {\rm H}^{-1} \sum_{i}\left\{ 
      \frac{\partial A }{\partial \bm{q}_{i}} \otimes \left({\rm H}^{-1} \bm{q}_{i} \right) 
      - \bm{p}_{i} \otimes \left( {\rm H}^{-1} \frac{\partial A }{\partial \bm{p}_{i}} \right)
    \right\}\\
    &+ \sum_{i}  \left\{ \left({\rm H}^{-1} \bm{q}_{i} \right) \otimes \frac{\partial A }{\partial \bm{q}_{i}} 
        - \left( {\rm H}^{-1} \frac{\partial A }{\partial \bm{p}_{i}} \right) \otimes \bm{p}_{i} \right\}{\rm H}^{-T}  \\
    &= \sum_{i} \left\{ H^{-1} \frac{\partial A }{\partial \bm{q}_{i}} \otimes H^{-1} \bm{q}_{i} - H^{-1} \bm{p}_{i} \otimes \left( {\rm H}^{-1} \frac{\partial A }{\partial \bm{p}_{i}} \right) \right. \\
    &+ \left. \left({\rm H}^{-1} \bm{q}_{i}\right) \otimes \frac{\partial A }{\partial \bm{q}_{i}} 
     {\rm H}^{-T}   
    - \left( {\rm H}^{-1} \frac{\partial A }{\partial \bm{p}_{i}} \right) \otimes \bm{p}_{i} {\rm H}^{-T} \right\} \\
    &= \sum_{i}  \left\{ H^{-1} \frac{\partial A }{\partial \bm{q}_{i}} \otimes \left( \bm{q}_{i} H^{-T}\right) 
    - \left( H^{-1} \bm{p}_{i} \right) \otimes \left( \frac{\partial A }{\partial \bm{p}_{i}} {\rm H}^{-T} \right)  \right. \\
    &+ \left. \left({\rm H}^{-1} \bm{q}_{i} \right) \otimes \frac{\partial A }{\partial \bm{q}_{i}} 
    {\rm H}^{-T} 
    - \left( {\rm H}^{-1} \frac{\partial A }{\partial \bm{p}_{i}} \right) \otimes\left( \bm{p}_{i} {\rm H}^{-T} \right) \right\}\\
    &= \sum_{i} {\rm H}^{-1}  \left\{ \frac{\partial A }{\partial \bm{q}_{i}} \otimes \bm{q}_{i}
    + \bm{q}_{i} \otimes \frac{\partial A }{\partial \bm{q}_{i}} 
    - \bm{p}_{i} \otimes \frac{\partial A }{\partial \bm{p}_{i}} 
    - \frac{\partial A }{\partial \bm{p}_{i}} \otimes \bm{p}_{i} \right\}{\rm H}^{-T}\\
    &:= {\rm H}^{-1} (\hat{D} A) {\rm H}^{-T} 
\end{align}
\begin{align}
\frac{1}{Z}\frac{\partial Z}{\partial H_{\alpha \beta}} &= -\frac{1}{k_{B}T} \left< \frac{\partial {\cal H}}{\partial H_{\alpha \beta}}\right>
\end{align}
\begin{align}
    \frac{\partial \langle A \rangle }{\partial H_{\alpha \beta}} &= \frac{\partial}{\partial H_{\alpha \beta}}\frac{1}{Z} \int d\bm{q} d\bm{p} A \exp(-{\cal H}/k_{B} T) \\
    &= -\frac{1}{Z^{2}} \frac{\partial Z}{\partial H_{\alpha \beta}} \int d\bm{q} d\bm{p} A \exp(-{\cal H}/k_{B} T)  
    + \frac{1}{Z}\int d\bm{q} d\bm{p} \frac{\partial A}{\partial H_{\alpha \beta}} \exp(-{\cal H}/k_{B} T) \\ 
    &+ \frac{1}{Z}\int d\bm{q} d\bm{p} A \frac{\partial }{\partial H_{\alpha \beta}} \exp(-{\cal H}/k_{B} T) \\
    &= -\frac{1}{Z}\left(-\frac{1}{k_{B}T}\right) \left< {\frac{\partial \cal  H}{\partial H_{\alpha \beta}}}\right> \left< A \right> +\left< \frac{\partial A}{\partial H_{\alpha \beta}}  \right> - \frac{1}{Z}\frac{1}{k_{B} T} \left< A \frac{\partial {\cal H}}{\partial H_{\alpha \beta}} \right> \\
    &=\left< \frac{\partial A}{\partial H_{\alpha \beta}}  \right> -\frac{1}{k_{B}T} \left(\left< A \frac{\partial {\cal H}}{\partial H_{\alpha \beta}} \right>  + \left< \frac{\partial {\cal H}}{\partial H_{\alpha \beta}}\right> \left< A \right> \right) \\
    &= \left< \frac{\partial A}{\partial H_{\alpha \beta}}  \right> -\frac{1}{k_{B}T} \left(\left< A \frac{\partial {\cal H}}{\partial H_{\alpha \beta}} \right>  + \left< \frac{\partial {\cal H}}{\partial H_{\alpha \beta}}\right> \left< A \right> \right)
\end{align}

\begin{align}
    \frac{\partial \langle A \rangle }{\partial H_{\alpha \beta}} &= \left< \frac{\partial A}{\partial H_{\alpha \beta}}  \right> -\frac{1}{k_{B}T} \left(\left< A \frac{\partial {\cal H}}{\partial H_{\alpha \beta}} \right>  + \left< \frac{\partial {\cal H}}{\partial H_{\alpha \beta}}\right> \left< A \right> \right)
\end{align}

\begin{align}
    \frac{\partial \left<A\right>}{\partial e_{\alpha \beta}} &= \frac{1}{2}\left({\rm H_{0}H^{-1}}\frac{\partial \left<A\right>}{\partial {\rm H}}{\rm H}_{0}^{T} + {\rm H_{0}}\frac{\partial \left<A\right>}{\partial {\rm H}^T}{\rm H^{-T}}{\rm H_{0}^{T}}\right)_{\alpha \beta} \\    
    &= \frac{1}{2}{\rm H_{0}} \left( {\rm H^{-1}}\frac{\partial \left<A\right>}{\partial {\rm H}} 
    + \frac{\partial \left<A\right>}{\partial {\rm H}^T}{\rm H^{-T}}\right) {\rm H_{0}^{T}} \\  
    &= \frac{1}{2}{\rm H_{0}} \left[ {\rm H^{-1}} \left\{ \left< \frac{\partial A}{\partial {\rm H}}  \right> -\frac{1}{k_{B}T} \left(\left< A \frac{\partial {\cal H}}{\partial {\rm H}} \right>  + \left< \frac{\partial {\cal H}}{\partial {\rm H}}\right> \left< A \right> \right) \right \}\right. \\
    &+ \left. \left\{ \left< \frac{\partial A}{\partial {\rm H}^{T}}  \right> -\frac{1}{k_{B}T} \left(\left< A \frac{\partial {\cal H}}{\partial {\rm H}^{T}} \right>  + \left< \frac{\partial {\cal H}}{\partial {\rm H}^{T}}\right> \left< A \right> \right) \right\} {\rm H^{-T}}\right] {\rm H_{0}^{T}} \\ 
\end{align}
\begin{align}
    F &= - k_{B} T  \ln Z \\
    \Omega \, t_{\alpha \beta} &= - \Omega \frac{\partial F}{\partial e_{\alpha \beta}} \\
    &= k_{B} T Z^{-1} \frac{\partial Z}{\partial e_{\alpha \beta}} \\
    &= \frac{1}{2} k_{B} T Z^{-1} \left({\rm H_{0}H^{-1}}\frac{\partial Z}{\partial {\rm H}}{\rm H}_{0}^{T} + {\rm H_{0}}\frac{\partial Z}{\partial {\rm H}^T}{\rm H^{-T}}{\rm H_{0}^{T}}\right)_{\alpha \beta} \\ 
    &= \frac{1}{2} k_{B} T \left\{ {\rm H_{0}H^{-1}}\left(-\frac{1}{k_{B}T} \left< \frac{\partial {\cal H}}{\partial {\rm H}}\right> \right){\rm H}_{0}^{T}  
    + {\rm H_{0}} \left(-\frac{1}{k_{B}T} \left< \frac{\partial {\cal H}}{\partial {\rm H}^{T}}\right> \right){\rm H^{-T}}{\rm H_{0}^{T}}\right\}_{\alpha \beta} \\  
    &= -\frac{1}{2} \sum_{i} \left\{ {\rm H_{0}H^{-1}}\left< \bm{V}_{i}  \otimes \left({\rm H}^{-1} \bm{q}_{i}\right) - \bm{p}_{i} \otimes \left( {\rm H}^{-1}\frac{\bm{p}_{i}}{m_{i}}  \right)  \right> {\rm H}_{0}^{T} \right.\\
    &+ \left. {\rm H_{0}} \left< \left({\rm H}^{-1} \bm{q}_{i}\right) \otimes \bm{V}_{i}
    -  \left( {\rm H}^{-1}\frac{\bm{p}_{i}}{m_{i}}  \right) \otimes \bm{p}_{i} \right> {\rm H^{-T}}{\rm H_{0}^{T}}\right\}_{\alpha \beta} \\ 
    &= -\frac{1}{2} \sum_{i} \left\{ {\rm H_{0}H^{-1}}\left< \bm{q}_{i} \otimes \bm{V}_{i} \right> {\rm H}^{-T} {\rm H}_{0}^{T} - {\rm H_{0} H^{-1}} \left<\bm{p}_{i} \otimes \frac{\bm{p}_{i}}{m_{i}}   \right> {\rm H}^{-T} {\rm H}_{0}^{T} \right.\\
    &+ \left. {\rm H_{0}}{\rm H}^{-1}  \left< \bm{V}_{i} \otimes  \bm{q}_{i} \right> {\rm H^{-T} H_{0}^{T}} 
    -   {\rm H_{0} H}^{-1} \left< \frac{\bm{p}_{i}}{m_{i}} \otimes \bm{p}_{i} \right> {\rm H^{-T}}{\rm H_{0}^{T}}\right\}_{\alpha \beta} \\
    &= -\frac{1}{2} \sum_{i} {\rm H_{0}H^{-1}} \left\{ \left< \bm{q}_{i} \otimes \bm{V}_{i} \right>  + \left< \bm{V}_{i} \otimes  \bm{q}_{i} \right> - \left<\bm{p}_{i} \otimes \frac{\bm{p}_{i}}{m_{i}}   \right>  
    -  \left< \frac{\bm{p}_{i}}{m_{i}} \otimes \bm{p}_{i} \right> \right\} {\rm H^{-T}}{\rm H_{0}^{T}} \\
    &:= -\frac{1}{2} \left< \hat{D}' {\cal H} \right> \\
    &:= \Omega \left< t_{\alpha \beta} \right>
\end{align}
\begin{align}
\frac{\partial {\cal H}}{\partial {\rm H}} &= \sum_{i} \left\{ \bm{V}_{i} \otimes \left({\rm H}^{-1} \bm{q}_{i}\right)
- \bm{p}_{i} \otimes \left( {\rm H}^{-1}\frac{\bm{p}_{i}}{m_{i}}  \right) \right\}     
\end{align}
\begin{align}
    C_{\alpha \beta \gamma \sigma} &= -\frac{\partial \left< t_{\alpha \beta} \right>}{\partial e_{\gamma \sigma}}\\
    &= -\frac{\partial }{\partial e} \left( - \frac{1}{2 \Omega} \left< \hat{D}' {\cal H} \right> \right) \\
    &= -\left< \hat{D}_{\gamma \sigma}' \hat{t}_{\alpha \beta} \right> + \frac{2 \Omega}{k_{B}T} \left( 
        \left< \hat{t}_{\alpha \beta} \hat{t}_{\gamma \sigma} \right>
         - \left< \hat{t}_{\alpha \beta} \right> \left< \hat{t}_{\sigma \gamma} \right>
    \right) \\
    &= \frac{1}{\Omega} \left< \hat{D}_{\gamma \sigma}' \left( \hat{D}_{\alpha \beta}' {\cal H} \right) \right> + \frac{2\Omega}{k_{B}T} \delta  \left(\hat{t}_{\alpha \beta} \hat{t}_{\gamma \sigma}\right)\\
    \left. C_{\alpha \beta \gamma \sigma} \right|_{e=0} &= \frac{2\Omega}{k_{B} T} \delta_{\hat{t}_{\alpha \beta} \hat{t}_{\gamma \sigma}}
     + \frac{1}{\Omega} \hat{S}_{\alpha \beta \gamma \sigma} \left< 
    \sum_{i, j} q_{i \gamma} V_{j\alpha, i\sigma} q_{j \beta}
     + \sum_{i} q_{i \gamma} V_{i \alpha} \delta_{\beta \sigma}
     + \sum_{i} \frac{2}{m_{i}} p_{i \gamma} p_{i \beta} \delta_{\alpha \sigma} 
    \right> \\
    &= \frac{2\Omega}{k_{B}T} \delta(\hat{t}_{\alpha \beta} \hat{t}_{\gamma \sigma}) 
    + \frac{1}{\Omega} \hat{S}_{\alpha \beta \gamma \sigma} \left\{ 
        \sum_{i} 2\delta_{\alpha \sigma} \delta_{\gamma \beta} k_{B} T
         + \left< \sum_{ij} q_{i \gamma} V_{j \alpha, i \sigma} q_{j \beta}  \right>
         + \left< \sum_{i} q_{i \gamma} V_{i \alpha} \delta_{\beta \sigma} \right>
         \right\} \\
    &= \frac{2\Omega}{k_{B}T} \delta(\hat{t}_{\alpha \beta} \hat{t}_{\gamma \sigma})
           + \frac{4Nk_{B}T}{\Omega}\left( \delta_{\alpha \sigma} \delta_{\beta \gamma} + \delta_{\beta \sigma}\delta_{\alpha \gamma} \right) \\
    &      + \frac{1}{\Omega} \hat{S}_{\alpha \beta} \hat{S}_{\gamma \sigma}\left\{ 
           \left< \sum_{ij} q_{i \gamma} V_{j \alpha, i \sigma} q_{j \beta}  \right>
          + \left< \sum_{i} q_{i \gamma} V_{i \alpha}\right> \delta_{\beta \sigma} 
          \right\} \\
\end{align}
\begin{align}
    \left< \frac{1}{m_{i}} p_{i \gamma} p_{i \beta} \right> &= \frac{1}{m_{i}} \frac{1}{Z} \int d\bm{q}^{3N} d\bm{p}^{3N} p_{i \gamma} p_{i \beta} \exp\left(-\frac{1}{k_{B}T}\left( \sum_{j}\frac{p_{j}^{2}}{2m_{j}} + V\left(\left\{\bm{q_{j}}\right\}\right)\right) \right) \\
    &= \delta_{\gamma \beta} \frac{1}{m_{i}} \frac{\sqrt{\pi}/2 (2m_{i}k_{B}T)^{3/2}}{\sqrt{\pi} (2 m_{i}k_{B}T)^{1/2}}\\
    &= \delta_{\gamma \beta} k_{B} T
\end{align}

\section{Appendix}
格子行列${\rm H}$の逆行列は逆格子ベクトルである。
\begin{align}
    \bm{a}_{i} \cdot \bm{b}_{j}&= \sum_{\alpha} a_{i \alpha} b_{j \alpha} = 2 \pi \delta_{ij} \\
    a_{11} b_{12} + a_{21}  b_{22} + a_{31} b_{32} &=
    a_{11} (a_{23}a_{31}-a_{21}a_{33}) + a_{21} (a_{33}a_{11}-a_{31}a_{13}) + a_{31}(a_{13}a_{21}-a_{11}a_{23})\\ 
    &= 0\\
    a_{11} b_{11} + a_{21}  b_{21} + a_{31} b_{31} &=
    \frac{a_{11} (a_{22}a_{33}-a_{23}a_{32}) + a_{21} (a_{32}a_{13}-a_{33}a_{12}) + a_{31}(a_{12}a_{23}-a_{13}a_{22})}{V/2\pi}\\ 
    &= \frac{a_{11} (a_{22}a_{33}-a_{23}a_{32}) + a_{12} (a_{23}a_{31} -a_{21} a_{33}) + a_{13}(a_{21} a_{32} - a_{22}a_{31})}{V/2\pi}\\
    &= \frac{\bf{a}_{1} \cdot \bf{a}_{2} \times \bf{a}_{3}}{V/2\pi}\\
    &= 2\pi\\
    \sum_{i} a_{i \alpha} b_{i \beta} &= a_{i \alpha} a_{i \gamma}\times a_{i \alpha} =2\pi \delta_{\alpha \beta} \\
    \rm{b}_{1} &= 2\pi \frac{ \rm{a}_{2} \times \rm{a}_{3}}{V} 
            = \frac{2\pi}{V} \begin{pmatrix} a_{22}a_{33} - a_{23}a_{32} \\ a_{23}a_{31} - a_{21}a_{33} \\ a_{21}a_{32} - a_{22}a_{31} \end{pmatrix} \\
    \rm{b}_{2} &= 2\pi \frac{ \rm{a}_{3} \times \rm{a}_{1}}{V} \\
    \rm{b}_{3} &= 2\pi \frac{ \rm{a}_{1} \times \rm{a}_{2}}{V} 
    \end{align}
    
    \begin{align}
    {\rm H}^{-1} &= \frac{1}{2\pi} \begin{pmatrix} \bm{b}_{1}^{T} \\ \bm{b}_{2}^{T} \\ \bm{b}_{3}^{T} \end{pmatrix} 
        = \frac{1}{2\pi}\begin{pmatrix} 
            b_{11} & b_{12} & b_{13} \\
            b_{21} & b_{22} & b_{23} \\
            b_{31} & b_{32} & b_{33} \\
        \end{pmatrix} \\
    \end{align}
    \begin{align}
        {\rm H^{-1} H} &=\frac{1}{2\pi} 
        \begin{pmatrix} 
            b_{11} & b_{12} & b_{13} \\
            b_{21} & b_{22} & b_{23} \\
            b_{31} & b_{32} & b_{33} \\
        \end{pmatrix} 
        \begin{pmatrix} 
            a_{11} & a_{21} & a_{31} \\
            a_{12} & a_{22} & a_{32} \\
            a_{13} & a_{23} & a_{33} \\
        \end{pmatrix} \\
        &=\frac{1}{2\pi} 
        \begin{pmatrix} 
            \sum_{\alpha} b_{1 \alpha} a_{1 \alpha} & \sum_{\alpha} b_{1 \alpha} a_{2 \alpha}  & \sum_{\alpha} b_{1 \alpha} a_{3 \alpha}  \\
            \sum_{\alpha} b_{2 \alpha} a_{1 \alpha} & \sum_{\alpha} b_{2 \alpha} a_{2 \alpha}  & \sum_{\alpha} b_{2 \alpha} a_{3 \alpha}  \\
            \sum_{\alpha} b_{3 \alpha} a_{1 \alpha} & \sum_{\alpha} b_{3 \alpha} a_{2 \alpha}  & \sum_{\alpha} b_{3 \alpha} a_{3 \alpha}  \\
        \end{pmatrix}\\ 
        &= {\rm I}
    \end{align}
    
\bibliography{main}
%
\end{document}
%